% load document class mhexsheet & select language as option (german or english)
\documentclass[german]{mhexsheet}

% set the parameters for the current exercise sheet with \exerciseSetup (key-value interface)
\exerciseSetup{
  lecture      = Linear and Network Optimization,
  lectureshort = Optimization, % optional short name for the footer
  semester     = WS 2013/14,
  deadline     = 19.4.2013,
  lecturer     = Stefan Ruzika,
  operator     = {Michael Helmling, David Willems},
  %homepage     = www.google.de,
  sheetnumber  = 12,
  %solution, % solution (or solution=true) enables output of solution environments. solution=false (default) disables it,
  % logo = \includegraphics[width=5cm]{image.png}, % can override the standard logo
  % logowidth = 7cm % ... and also its width
}
\usepackage{mh_math}

\begin{document}
\maketitle

\begin{exercise}
With \verb|\begin{enumerate}[columns=<n>]| you can create multicolumn subexercise lists for small exercises:
\begin{enumerate}[columns=3]
  \item $1+2\alpha xyz \sum\mathbb{R}\mathcal{P}\int_5^x$
  \item $2+3$
  \item $3+4$
  \item $4+5$
  \item $5+6$
\end{enumerate}
If you ever feel the need, a manual column break can be inserted with \verb|\columnbreak|.
\end{exercise}

\begin{solution}
Solutions are output if the option \verb|solution| (or \verb|solution=true|) is passed to \verb|\exerciseSetup|.
\begin{enumerate}
\item 3
\end{enumerate}
\end{solution}

\begin{exercise}[title=Lemma of Zorn,points=100]
Optional titles and point numbers can be assigned to exercises:\\
\verb!\begin{exercise}[title=Lemma of Zorn,points=100]!
\end{exercise}
Predefined captions (like \enquote{Exercise} / \enquote{Aufgabe}) can be changed with \verb|mhexCaption|. For example,
\begin{verbatim}
\mhexCaption{exercise}{Übungsaufgabe}
\end{verbatim}
changes the caption used for the exercise environment:
\mhexCaption{exercise}{Übungsaufgabe}
\begin{exercise}
Test
\end{exercise}
You should however avoid this because it breaks the uniform appearence of the exercise sheets.
\end{document}