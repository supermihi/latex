\documentclass[11pt,german,a4paper,parskip=half-]{scrartcl}
% Übungsblatt erstellen: xelatex uebungX.tex
% Lösungsblatt erstellen: xelatex -jobname loesungX uebungX.tex
\usepackage[modern]{agopt_ex}
\usepackage{mh_math}
\usepackage{listings}
\usepackage{url}
\usepackage{xspace}

\renewcommand{\u}[1]{\ensuremath{\underline{#1}}} % für Vektoren in Hamacher-Schreibweise
\hyphenation{equi-va-lent}
%\usepackage{enumerate}
\Sheetnumber{5}
\Deadline{26.05.2011 13:45 Uhr, Erdgeschoss Geb. 48}
\Lecture{PraMa: Lineare und Netzwerkoptimierung}
\Semester{SS 2011}
\Homepage{http://optimierung.mathematik.uni-kl.de/teaching/ss11/prama.html}
\LectureShort{PraMa Optimierung}
\Lecturer{Dr.\ Florentine Bunke, Jun.-Prof.\ Dr.\ Stefan Ruzika}
\Operator{Michael Helmling, Florian Seipp}

\sloppy
\begin{document}
\maketitle

\begin{exercise}{4}
 Gegeben sei das folgende lineare Programm:
\[\text{(LP)}\quad\begin{array}{rcrcrcrcrcr}
\min && 2x_{1} &+& 3x_{2} &+& 5x_{3} &+& 6x_{4} \\
\text{s.t.} && x_{1} &+& 2x_{2} &+& 3x_{3} &+& x_{4} &\geq& 2\\
&-& x_{1} &+& x_{2} &-& x_{3} &+& 3x_{4} &\leq& -3 \\
&&&&& \multicolumn{4}{r}{x_{1},x_{2},x_{3},x_{4}}&\geq& 0
\end{array}\]
\begin{enumerate}
 \item Stellen Sie das zu (LP) duale Problem (DLP) auf. Lösen Sie dieses mit einem Verfahren Ihrer Wahl (grafisch, Tableau, GLPK, Matlab).
  Benutzen Sie dann den Satz vom komplementären Schlupf, um eine primale Optimallösung für (LP) zu finden.
 \item Lösen Sie (LP) mit dem dualen Simplex (mit den Schlupfvariablen als Startbasis).
\end{enumerate}
{\it Hinweis:} Wenn Sie richtig rechnen, sollte in beiden Teilaufgaben das selbe Ergebnis herauskommen!
\end{exercise}
\begin{solution}[2+2 Punkte]
 \begin{enumerate}
  \item Zum Dualisieren kann man die 2. Ungleichung mit $(-1)$ multiplizieren, dann erhält man als duale Variablen $\pi_1, \pi_2 \geq 0$ und (DLP) ist
\begin{align*}
 \max\;&2\pi_1 + 3 \pi_2\\
\st\quad&\pi_1+\pi_2 \leq 2\\
&2\pi_1 -\pi_2 \leq 3\\
&3 \pi_1 + \pi_2 \leq 5\\
&\pi_1 - 3\pi_2 \leq 6\\
&\u \pi \geq \u 0
\end{align*}
Die Optimallösung von (DLP) ist $\hat{\u{\pi}} = (0,2)^T$ mit Zielfunktionswert $6$. Also hat auch (LP) eine optimale Lösung $\hat{\u x}$ mit ZFW $6$. Diese bestimmen wir nun mit dem Satz vom komplementären
Schlupf: $\hat{\u \pi}$ erfüllt nur die erste Ungleichung von (DLP) mit Gleichheit, also kann nur $\hat x_1 \neq 0$ sein. Da umgekehrt $\hat\pi_2 \neq 0$ ist, muss die zweite Ungleichung von (LP) mit Gleichheit
erfüllt sein, also gilt $\hat x_1 = 3$ und die Optimallösung ist $\u{\hat x} = (3,0,0,0)$.
\item Jetzt empfiehlt es sich, die erste Gleichung von (LP) mit $(-1)$ zu multiplizieren. Dann ist das Starttableau
\[\begin{array}{rrrrrr|r}
   2&3&5&6&0&0&0\\\hline
   -1&-2&-3&-1&1&0&-2\\
   -1&1&-1&3&0&1&-3
  \end{array}\]
und nach einer Pivotoperation (mit dem Element $(t_{2,1})$) erhalten wir das optimale Tableau
\[\begin{array}{rrrrrr|r}
   0&5&3&0&0&2&-6\\\hline
   0&-3&-2&-4&1&-1&5\\
   1&-1&1&-3&0&-1&3
  \end{array}\]
 \end{enumerate}
\end{solution}


\begin{exercise}{4}
Sei $A \in \R^{m\times n}$ und $\u b \in \R^m$. Das \emph{Lemma von Farkas} besagt, dass genau eines der beiden
folgenden Systeme eine Lösung ($\u x \in \R^n$ bzw. $\u y \in \R^m$) besitzt:
\[ \text{(1):}\; A \u x = \u b,\; \u x \geq \u 0 \qquad \text{(2):}\;\u b^T \u y < 0,\; A^T \u y  \geq \u 0\]
\begin{enumerate}
 \item Beweisen Sie das Lemma von Farkas, indem Sie ein geeignetes LP aufstellen und den starken Dualitätssatz anwenden.
 \item Finden Sie eine geometrische Interpretation des Lemmas. Verwenden Sie dazu folgende Definition: Seien $\u a^1,\dotsc, \u a^k \in \R^n$ gegeben,
 dann heißt
\[ \op{cone}(\u a^1,\dotsc,\u a^k) := \left\{ \sum_{i=0}^k \lambda_i \u a^i:\, \lambda_i \geq 0\text{ für }i=1,\dotsc,n\right\}\]
der von $\u a^1,\dotsc,\u a^k$ aufgespannte \emph{Kegel}. Skizzieren Sie für $\u y \in \R^2$ ein Beispiel.\\
{\it Hinweis:} Was bedeuten die beiden Aussagen für die Lage von $\u b$ zu den \emph{Spaltenvektoren} von $A$?
\end{enumerate}
\end{exercise}

\begin{solution}
\begin{enumerate}
 \item Betrachte das Problem
\begin{align*}
 \text{(LP)}\quad\max\;&\u 0\ \u x\\
\st\quad&A\u x  = \u b\\
&\u x\geq \u 0
\end{align*}
mit dem dualen Problem
\begin{align*}
 \text{(DLP)}\quad\min\;&\u b^T\u y\\
\st\quad&A^T\u y \geq \u 0
\end{align*}
Angenommen (1) ist lösbar, dann ist (LP) zulässig und (da die Zielfunktion immer $0$ ist) auch beschränkt, hat also eine Optimallösung mit ZFW $0$. Nach dem Dualitätssatz ist dann auch (DLP) optimal lösbar
mit ZFW $0$, also existiert kein $\u y$ das (2) erfüllt.

Sei umgekehrt (2) nicht lösbar. Damit ist $\u y = \u 0$ eine Optimallösung für (DLP). Nach dem Dualitätssatz hat dann auch (LP) eine Optimallösung, d.h. (1) ist lösbar.

Wir haben also gezeigt: (1) lösbar $\Llra$ (2) nicht lösbar; mit anderen Worten: Genau eines der Systeme ist lösbar.
\item Eine Lösung $\u x$ von (1) bedeutet, dass $\u b$ im durch die Spalten von $A$ aufgespannten Kegel liegt: $\u b = \sum_{i=1}^n x_i A_{•,i}$ mit $x_i \geq 0$ für $i=1,\dotsc,n$ ($\u x$ entspricht also
dem $\u \lambda$ in der Definition des Kegels).

Wenn umgekehrt (2) lösbar ist, gibt es eine Hyperebene, definiert durch $H_{\u y} := \{ \u z \in \R^m:\,  \u z^T \u y = 0\}$, die $\u b$ von den Spalten von $A$ trennt: $\u b$ liegt auf der einen Seite ($\u b^T\u y < 0$) und alle Spalten von $A$ auf der anderen ($(A^T \u y)_j = A_{•,j}^T \u y \geq 0$). Ein Bild für $n=4,m=2$:
\begin{center}
\begin{tikzpicture}[every node/.style={fill=red,draw,circle,inner sep=0mm,minimum size=1.5mm},scale=.8]

 \draw[->] (-1,0) -- (5,0);
 \draw[->] (0,-1) -- (0,5);
 \node[fill=blue,label=left:{$\u b$}] at (-.5,3) {};
 \node[label=right:{$A_{•,1}$}] at (1,4) {};
 \node[label=right:{$A_{•,2}$}] at (1.3,1) {};
 \node[label=right:{$A_{•,3}$}] at (3,1) {};
 \node[label=right:{$A_{•,4}$}] at (2,-1) {};
 \draw[thick,gray,dashed] (0.5,5) -- node[right,fill=none,draw=none] {$\u z^T \u y = 0$} (-.15, -1.5);
\end{tikzpicture}\qquad
\begin{tikzpicture}[every node/.style={fill=red,draw,circle,inner sep=0mm,minimum size=1.5mm},scale=.8]

 \draw[->] (-1,0) -- (5,0);
 \draw[->] (0,-1) -- (0,5);
\draw[thick,gray,dashed] (0,0) -- (1.3,1) -- (4.3,2); 
\node[fill=blue,label=right:{$\u b = 1\cdot A_{•,2} + 1\cdot A_{•,3}$}] at (4.3,2) {}; 
 \node[label=right:{$A_{•,1}$}] at (1,4) {};
 \node[label=right:{$A_{•,2}$}] at (1.3,1) {};
 \node[label=right:{$A_{•,3}$}] at (3,1) {};
 \node[label=right:{$A_{•,4}$}] at (2,-1) {};
\end{tikzpicture}
\end{center}
\end{enumerate}
\end{solution}

\begin{exercise}{4}
Auch wenn der Simplex-Algorithmus in den meisten Fällen sehr schnell ist und deshalb auch in der Praxis eingesetzt wird, ist seine Worst-Case-Laufzeit exponentiell.
Auf diesem und dem nächsten Übungsblatt soll dies anhand des sogenannten \emph{Klee-Minty-Würfels} gezeigt werden. Dabei handelt es sich um einen leicht verzerrten
Hyperwürfel mit $2^n$ Eckpunkten, die im ungünstigsten Fall alle vom Simplex durchlaufen werden.

Es sei $0 < \varepsilon < \frac12$ und $n \in \N$. Wir betrachten das LP
\[\max\,\left\{x_n\,|\,\varepsilon x_{i-1} \leq x_i \leq 1-\varepsilon x_{i-1}\text{ für }i=1,\dotsc,n\text{ und } \u x \geq \u 0\right\}\tk\]
wobei $x_0=1$ festgesetzt (keine Variable) ist. Beachten Sie, dass der zulässige Bereich bei $\varepsilon=0$ genau der Einheitswürfel wäre.
\begin{enumerate}
 \item Bringen Sie das LP in Standardform, indem Sie für die Ungleichungen auf der linken Seite Schlupfvariablen $r_i$, für die auf der rechten Seite Schlupfvariablen $s_i$ einführen.
 \item Zeigen Sie, dass jede zulässige Basis alle $x_i$ und für jedes $i\in\{1,\dotsc,n\}$ entweder $r_i$ oder $s_i$ enthält. Ist das Problem degeneriert?\\
{\it Hinweis:} Verwenden Sie, dass eine Variable, die nicht $0$ sein kann, in jeder Basis enthalten sein muss, und dass die Basis so viele Elemente wie das LP Zeilen haben muss.
 \item Die vorherige Teilaufgabe zeigt, dass in jeder Basislösung $x$ jedes $x_i$ entweder den minimalen $(r_i=0)$ oder maximalen $(s_i=0)$ Wert annimmt und dass die Basis dadurch bereits
vollständig definiert ist – es existiert also eine 1-zu-1-Beziehung zwischen den Basislösungen und den Teilmengen $L \subseteq N := \{1,\dotsc,n\}$ durch die Festlegung $i \in L \Llra s_i=0$. Wir
erhalten dann für jedes solche $L$ durch
\[J^L := \{x_1,\dotsc,x_n\} \cup \{r_i:\, i \in L\} \cup \{s_i:\, i \in N \setminus L\}\]
eine zulässige Basis. Sei $x^L$ die zugehörige Basislösung. Zeigen Sie, dass für $L, L' \subseteq N$ mit $n \in L$ und $n \notin L'$ gilt: $x_n^{L'} < x_n^L$, d.h. dass jede Basis, die $r_n$ enthält,
einen echt größeren Zielfunktionswert hat als alle Basen, die $r_n$ nicht enthalten. Zeigen Sie weiter, dass für $L' \definedAs L\setminus\{n\}$ gilt: $x_n^{L'} = 1-x_n^L$.
\end{enumerate}
\end{exercise}

\begin{solution}
\begin{enumerate}
 \item Standardform:
\begin{align*}
 -\min\;&-x_n\\
\st\quad&x_1 + s_1 = 1-\varepsilon\\
&x_1-r_1 = \varepsilon\\
&\varepsilon x_{i-1} + x_i + s_i = 1&i=2,\dotsc,n\\
&-\varepsilon x_{i-1} + x_i - r_i = 0&i=2,\dotsc,n\\
&r_i,s_i \geq 0&i=1,\dotsc,n
\end{align*}
\item Wir zeigen zunächst per Induktion, dass alle $x_i>0$ sind. Induktionsanfang ($n=1$): $x_1 > \varepsilon x_0 = \varepsilon > 0$. Induktionsschluss ($n-1\rightarrow n$):
$x_n$ erfüllt $\varepsilon x_{n-1} \leq x_n$, also $x_n > 0$ da $\varepsilon >0$ und $x_{n-1} > 0$ nach Induktion.

Nun zeigen wir (ohne Induktion), dass für kein $i$ sowohl $r_i=0$ als auch $s_i=0$ sein kann. Für $i=1$: Falls $s_1=r_1=0$ lassen sich die ersten zwei Gleichungen in der Standardform zusammenfassen zu
$\varepsilon = 1-\varepsilon$, was wegen $\varepsilon < \frac12$ ein Widerspruch ist. Für $i>1$ kann man unter der Annahme $s_i=r_i=0$ analog die dritte und vierte Gleichung für dieses $i$ zusammenfassen
und erhält $1-\varepsilon x_{i-1} = \varepsilon x_{i-1}$ bzw. $\frac12 = \varepsilon x_{i-1}$, was wegen $\varepsilon < \frac12$ und $x_{i-1} < 1$ (was offensichtlich gilt) ein Widerspruch ist.
Also sind alle $x_i$ und jeweils mindestens eines von $\{r_i,s_i\}$ echt größer als $0$. Da das LP $2n$ Nebenbedingungen hat, enthält jede Basis $2n$ Variablen, also können auch für kein $i$ sowohl 
$r_i$ als auch $s_i$ in der Basis enthalten sein $\folgt$ jede Basis enthält alle $x_i$ und \emph{genau} eine der Variablen $\{r_i,s_i\}$ für jedes $i=1,\dotsc,n$. Das Problem ist nicht degeneriert: wir haben
gezeigt dass alle Basisvariablen $>0$ sein müssen.
\item Sei $n \in L \folgt r^L_n > 0$ und $s^L_n = 0 \folgt x_n^L = 1-\underbrace{\varepsilon x_{n-1}^L}_{<\frac12} > \frac12$. (1)\\
Sei $n \notin L' \folgt r_n^{L'} = 0$ und $s_n^{L'} > 0 \folgt x_n^{L'} = \underbrace{\varepsilon x_{n-1}^{L'}}_{<\frac12} < \frac12$. (2)\\
$\folgt x_n^{L'} < x_n^{L}$.

Sei nun $L' = L\setminus \{n\}$. Dann ist
\[x_n^{L'} \stackrel{\text{(2)}}{=} \varepsilon x_{n-1}^{L'} = \varepsilon x_{n-1}^L \stackrel{\text{(1)}}{=} 1-x_n^L\tk\]
wobei die mittlere Gleichung gilt da $L$ und $L'$ eingeschränkt auf $\{1,\dotsc,n-1\}$ übereinstimmen.

\end{enumerate}
\end{solution}

\lstset{basicstyle=\small\ttfamily,keywordstyle=\bfseries,frame=single,language=Matlab,numbers=left}
\begin{exercise}[Praktische Aufgabe]{4}
\begin{enumerate}
 \item 
Implementieren Sie eine Matlab-/Octave-Funktion
\begin{center}\lstinline!function [A_Ph1, b_Ph1, c_Ph1, B_Ph1 ] = InitPhase1 (A, b, c)!\end{center}
die die Eingabedaten eines LP in Standardform so transformiert (also ggf. künstliche Variablen einfügt, die Zielfunktion ersetzt etc.), dass Phase 1 des Simplex direkt angewendet werden kann. Dabei ist ${\tt B\_Ph1} \in \N^m$  der Indexvektor der Startbasis
für Phase 1.
\item Benutzen Sie Ihre bisherige Programme, um eine Funktion
\begin{center}\lstinline!function [EndTab, x_opt, z_opt] = TwoPhaseMethod(A,b,c, rule)!\end{center}
zu implementieren, die jedes beliebige LP in Standardform löst. Der Parameter \lstinline!rule! soll sich so verhalten wie zuvor. Das Verhalten bei unbeschränkten LPs soll dem der Funktion
\lstinline!simplex! auf dem letzten Blatt entsprechen. Ist das LP unzulässig, soll \lstinline!x_opt = z_opt = []! gelten.\\
{\it Hinweis:} Ändern Sie vorher Ihre Funktion \lstinline!simplex! so ab, dass auch die optimale Basis \lstinline!B_opt! ausgegeben wird.
\end{enumerate}

\end{exercise}

\begin{solution}
\lstset{basicstyle=\tiny\ttfamily}
 \begin{enumerate}
  \item~\hfill \lstinputlisting{InitPhase1.m}
  \item~\hfill \lstinputlisting{TwoPhaseMethod.m}
 \end{enumerate}

\end{solution}


\end{document}
